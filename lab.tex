\documentclass{article}
\usepackage[utf8]{inputenc}
\usepackage[polish]{babel}
\usepackage[T1]{fontenc}
\usepackage{graphicx} 
\graphicspath{{images/}} 
\usepackage{booktabs, makecell, longtable}
\usepackage{needspace}
\usepackage{pgfplots}
%\usepackage[font=scriptsize]{caption} 
\usepackage{hyperref}


\title{Spadek swobodny}
\author{Igor Gołębiowski}
\date{December 2022}

\begin{document}

\maketitle

\section{Wprowadzenie teoretyczne}

\begin {equation}
s=\frac{gt^2}{2}
\end {equation}

\section{Przebieg eksperymentu}

\begin{figure}[h]

    \centering
    \includegraphics[width=0.5\textwidth]{images/rys0010.jpg}
    \caption{Układ odniesienia związany z podłożem.}
    \label{fig:rysunek1}

\end{figure}

    Jak widać na Rysunek \ref{fig:rysunek1}. […] Lorem ipsum dolor sit amet, consectetur adipiscing elit.

    \newpage
    
\section{Wyniki pomiarów}

  
\begin{center}
\begin{longtable}{|r|r|}
\hline
\textbf{czas[s]} & \textbf{droga[m]} \\
\hline
\endfirsthead
\multicolumn{2}{r}%
{} \\
\hline 
\textbf{czas[s]} & \textbf{droga[m]} \\
\hline
\endhead
\endfoot
\hline
\caption{Pomiary drogi i czasu w spadku swobodnym.}
\endlastfoot
0.1  &2.10 \\ \hline
0.2  &0.72 \\ \hline
0.3  &3.37 \\ \hline
0.4  &-2.42 \\ \hline
0.5  &1.28 \\ \hline
0.6  &-4.55 \\ \hline
0.7 &3.61 \\ \hline
0.8 &0.37 \\ \hline
0.9 &1.39 \\ \hline
1 &-0.87 \\ \hline
1.1 &3.70 \\ \hline
1.2 &3.57 \\ \hline
1.3 &6.24 \\ \hline
1.4 &11.07 \\ \hline
1.5 &5.71 \\ \hline
1.6 &10.41 \\ \hline
1.7 &17.49 \\ \hline
1.8 &28.09 \\ \hline
1.9 &10.46 \\ \hline
2 &22.71 \\ \hline
2.1 &10.73 \\ \hline
2.2 &26.81 \\ \hline
2.3 &27.38 \\ \hline
2.4 &26.04 \\ \hline
2.5 &36.09 \\ \hline
2.6 &35.11 \\ \hline
2.7 &41.39 \\ \hline
2.8 &51.48 \\ \hline
2.9 &40.17 \\ \hline
3 &50.47 \\ \hline
3.1 &56.18 \\ \hline
3.2 &43.53 \\ \hline
3.3 &52.09 \\ \hline
3.4 &53.14 \\ \hline
3.5 &60.04 \\ \hline
3.6 &65.69 \\ \hline
3.7 &64.20 \\ \hline
3.8 &71.27 \\ \hline
3.9 &76.51 \\ \hline
4 &77.67 \\ \hline
4.1 &76.56 \\ \hline
4.2 &87.65 \\ \hline
4.3 &91.20 \\ \hline
4.4 &98.04 \\ \hline
4.5 &101.50 \\ \hline
4.6 &108.87 \\ \hline
4.7 &120.35 \\ \hline
4.8 &111.50 \\ \hline
4.9 &115.06 \\ \hline
5 &118.79 \\ \hline
5.1 &137.21 \\ \hline
5.2 &131.92 \\ \hline
5.3 &137.15 \\ \hline
5.4 &143.13 \\ \hline
5.5 &150.23 \\ \hline
5.6 &155.95 \\ \hline
5.7 &157.36 \\ \hline
5.8 &158.97 \\ \hline
5.9 &161.51 \\ \hline
6 &178.87 \\ \hline
6.1 &179.55 \\ \hline
6.2 &185.25 \\ \hline
6.3 &197.05 \\ \hline
6.4 &197.62 \\ \hline
6.5 &206.50 \\ \hline
6.6 &211.95 \\ \hline
6.7 &216.29 \\ \hline
6.8 &227.10 \\ \hline
6.9 &230.50 \\ \hline
7 &234.99 \\ \hline
7.1 &251.13 \\ \hline
7.2 &250.85 \\ \hline
7.3 &263.66 \\ \hline
7.4 &271.97 \\ \hline
7.5 &274.83 \\ \hline
7.6 &284.23 \\ \hline
7.7 &287.78 \\ \hline
7.8 &301.11 \\ \hline
7.9 &311.52 \\ \hline
8 &312.37 \\ \hline
8.1 &327.10 \\ \hline
8.2 &336.05 \\ \hline
8.3 &345.25 \\ \hline
8.4 &343.64 \\ \hline
8.5 &359.45 \\ \hline
8.6 &359.21 \\ \hline
8.7 &360.50 \\ \hline
8.8 &377.88 \\ \hline
8.9 &390.69 \\ \hline
9 &400.54 \\ \hline
9.1 &400.62 \\ \hline
9.2 &412.81 \\ \hline
9.3 &417.95 \\ \hline
9.4 &427.43 \\ \hline
9.5 &439.56 \\ \hline
9.6 &452.64 \\ \hline
9.7 &467.05 \\ \hline
9.8 &467.05 \\ \hline
9.9 &482.22 \\ \hline
10 &485.93  \\ \hline


\end{longtable}
\end{center}

\begin{center}
\begin{tikzpicture}
\begin{axis}[
    title={Wykres pomiarów drogi od czasu w spadku swobodnym},
    xlabel={t [s]},
    ylabel={s [m]},
    xmin=0, xmax=10,
    ymin=0, ymax=500,
    xtick={0,1,2,3,4,5,6,7,8,9,10},
    ytick distance=40,
    legend pos=north west,
    ymajorgrids=true,
    grid style=dashed,
    enlargelimits=false,
    height=10cm,
    width=12cm,
]
\addplot[
mesh,
domain=-10:10,
smooth,
thin,
orange,
]
{(9.81*(x^2))/2};


\addplot+[only marks, mark size=0.7pt, color=blue, each nth point={5}]
  plot [error bars/.cd, y dir=both, y explicit]
  table [y error minus=error] {toto.csv};

  
\end{axis}
\end{tikzpicture}
\end{center}

\section{Wnioski}

Lorem ipsum dolor sit amet, consectetur adipiscing elit. Proin fringilla magna tellus, eu porta justo porttitor eget. Interdum et malesuada fames ac ante ipsum primis in faucibus. Donec volutpat vehicula enim sed pellentesque. Donec massa augue, imperdiet eu maximus vel, accumsan vitae lectus. Praesent eget tristique massa. Aliquam viverra commodo varius. 

\tableofcontents
\hypersetup{linktocpage}

\end{document}


